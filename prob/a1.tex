\documentclass[a4paper]{article}
\usepackage{ucs}  % unicode
\usepackage[utf8x]{inputenc}
% \usepackage[T2A]{fontenc}
% \usepackage[bulgarian]{babel}
\usepackage{graphicx}
% \usepackage{fancyhdr}
% \usepackage{lastpage}
\usepackage{listings}
\usepackage{amsfonts}
\usepackage{amsmath}
% \usepackage{fancyvrb}
% \usepackage[usenames,dvipsnames]{color}
% \setlength{\headheight}{12.51453pt}

%\pagestyle{fancy}
%\fancyhead{}
%\fancyfoot{}

% \cfoot{\thepage\ от \pageref{LastPage}}

% \addto\captionsbulgarian{%
%   \def\abstractname{%
%     Цел на проекта} %\cyr\CYRA\cyrs\cyrt\cyrr\cyra\cyrk\cyrt}}%
% }

% Custom defines:
\def\rbc{foo bar}
\def\dc{bar baz}

% TODO remove colorlinks before printing
% \usepackage[unicode,colorlinks]{hyperref}   % this has to be the _last_ command in the preambule, or else - no work
% \hypersetup{urlcolor=blue}
% \hypersetup{citecolor=PineGreen}

\begin{document}

% 1.Подробно математическо описание на алгоритъма
% 2. Описание на работата на програмата с примери, разпечатка на кода,
% подробен user guide. Кодът трябва да е написан ясно и четливо, с подходящи коментари
% 3. Кривите да се чертаят като се използва алгоритъма на de Casteljau

% \title{\rbc}
% \author{
% Зорница Атанасова Костадинова, 4 курс, КН, фн: 80227
% }
% \date{\today}
% \maketitle
% 
% %\includegraphics[scale=0.1]{drop}
% 
% \begin{abstract}
% Настоящият документ е курсова работа към проекта ``\rbc'' по предмета ``Компютърно геометрично моделиране''. Описан е математическият алгоритъм за пресмятане на кривата. Обяснена е програмната реализация на проекта.
% \end{abstract}
% \newpage
% 
% \setcounter{tocdepth}{2}
% \tableofcontents
% \newpage

\title{The probabilistic method and randomized algorithms - Exercise 1 (rev. 2)}
\author{Iskren Ivov Chernev}
\maketitle

% \begin{enumerate}
%   \item \em{}Solution\em \\
%   Let us assume that $ n = 3k $.
%   Then it is easy to follow that
%   $$
%     k = \log \sqrt{n}
%   $$
%   which solves everything.
%   \item \em{}Solution\em
% \end{enumerate}

\section{Solution}

\def\sm{\mathrm{sm}}
\def\P{\mathbb{P}}
\def\E{\mathbb{E}}
\def\irv{i.r.v.\;}
\def\rv{r.v.\;}
\def\FMM{F.M.M.\;}
\def\endproof{$\square$}
\def\Var{\mathrm{Var}}
% \newcommand{\irv}{i.r.v\;}
\newcommand{\bfrac}[2] {\left(\frac{#1}{#2}\right)}
\newcommand{\B}[1] {\left(#1\right)}
% \newcommand{\Binom}[2] {\B{

Let $ X_{\sm} $ \irv showing weather the submatrix $ \sm $ is constant.

$$
  \P(X_{\sm} = 1) = 2 \left(\frac{1}{2}\right)^{j^2}
$$

Let $ X $ is a \rv showing the number of constant submatrixes of size $ j $.

$$
  X = \sum_{\sm} X_{\sm}
$$

Lets calculate $ \E X $:

\begin{eqnarray*}
  \E X &=& \sum_{\sm} \E X_{\sm} \\
       &=& \sum_{\sm} \P(X_{\sm} = 1) \\
       &=& \binom{n}{j}^2 2 \left(\frac{1}{2}\right)^{j^2} \\
\end{eqnarray*}

Lets get an upper bound on $ \E X $:

\begin{eqnarray*}
  \E X &=& \binom{n}{j}^2 2 \left(\frac{1}{2}\right)^{j^2} \\
       &=& \binom{\lfloor 2^{j/2} \rfloor}{j}^2 2 \left(\frac{1}{2}\right)^{j^2} \\
       &\le& \binom{2^{j/2}}{j}^2 2 \left(\frac{1}{2}\right)^{j^2} \\
       &\le& \left(\frac{(2^{j/2})^j}{j!}\right)^2 2\left(\frac{1}{2}\right)^{j^2} \\
       &=& \frac{2^{\frac{j}{2} 2j}}{(j!)^2} \frac{2}{2^{j^2}} \\
       &=& \frac{2}{(j!)^2} \\
       &\le& \frac{1}{2}
\end{eqnarray*}

Now, using the \FMM we conclude, that exists a matrix, for which the count $ X $ of constant submatrices with size $ j $ is less than $ \frac{1}{2} $. But $ X \in \mathbb{N} $ so $ X = 0 $. Which means there are no constant submatrices. \endproof

\section{Solution}

I'll show that $ n^\star = \frac{k}{\sqrt{2}e}2^{\frac{k}{2}} $. Since $ k = O(\log n) = o(\sqrt n) $ then we can use $ \binom{n}{k} \sim \frac{n^k}{k!} $

\begin{eqnarray*}
  \binom{n}{k} 2^{1-\binom{k}{2}}
  &\sim& \frac{\left(\frac{k}{\sqrt{2}e} 2^{\frac{k}{2}}\right)^k 2}{k! 2^{\frac{k^2}{2}-\frac{k}{2}}} \\
  &\sim& \frac{
              \left(\frac{k}{e}\right)^k 2^\frac{k^2}{2}
          }{
              2^{\frac{k}{2}} \sqrt{2\pi k} \left(\frac{k}{e}\right)^k 2^{\frac{k^2}{2}-\frac{k}{2}}
          } \\
  &=& \frac{1}{\sqrt{2\pi k}} \longrightarrow 0
\end{eqnarray*}

Now we'll try a slightly bigger function $ n^+ = (1 + \epsilon) \frac{k}{\sqrt{2}e}2^{\frac{k}{2}} $:

\begin{eqnarray*}
  \binom{n^+}{k} 2^{1-\binom{k}{2}}
  &\sim& \frac{\left((1 + \epsilon)\frac{k}{\sqrt{2}e} 2^{\frac{k}{2}}\right)^k 2}{k! 2^{\frac{k^2}{2}-\frac{k}{2}}} \\
  &\sim& (1 + \epsilon)^k \frac{
              \left(\frac{k}{e}\right)^k 2^\frac{k^2}{2}
          }{
              2^{\frac{k}{2}} \sqrt{2\pi k} \left(\frac{k}{e}\right)^k 2^{\frac{k^2}{2}-\frac{k}{2}}
          } \\
  &=& \frac{(1+\epsilon)^k}{\sqrt{2\pi k}} \longrightarrow \infty
\end{eqnarray*}

Hence $ n^\star \sim \frac{k}{\sqrt{2}e}2^{\frac{k}{2}} $. \endproof

\section{Solution}

We choose randomly $ z $ points out of $ n^2 $.

Let $ X_{ij} $ is \irv that the points $ i, j $ are such that their middle
point exists. To calculate the probability we'll use the fact, that $ i,
j $ can have middle point only if the distances on $ x $ and $ y $ axes are
both even numbers -- denoted by $ \mathrm{even}(\Delta i,j) $ (otherwise the
middle point doesn't have integer coordinates). $ \P(\mathrm{even}(\Delta i,j))
$ is $ \bfrac{1}{2}^2 $ for $ n $ even and $ \bfrac{n^2+1}{2n^2}^2 $ for
$ n $ odd.  Both of these tend to $ \frac{1}{4} $ for $ n \longrightarrow
\infty $. Also the probability that there is a point at any fixed position is
$ \frac{z-2}{n^2-2} $.
$$
  \P(X_{ij} = 1) = \frac{1}{4} \frac{z - 2}{n^2 - 2}
$$

Let $ X $ is the number of pair of points, whose middle one exists: $$ X = \sum_{ij} X_{ij} $$

Lets calculate $ \E X $
$$
  \E X = \sum_{ij} \P(X_{ij} = 1) = \binom{z}{2} \frac{1}{4} \frac{z-2}{n^2-2} = \frac{[z]^3}{8(n^2-2)}
$$

From the \FMM we conclude that there exists assignment of $ z $ points out of
$ n^2 $ in which $ X \le \E X $. If we remove that many points from $ z $ all
the other pairs won't have middle point. So the left number of points without
middles is at least $$ A = z - \frac{[z]^3}{8(n^2-2)} $$ It is obvious, that
if $ z = \Theta(n) $ than $ A = \Theta(n) $. We want to find a constant $ k $ in $ z = kn $ such that $ A \sim k'n $ yields the largest $ k' $. For that we only need to calculate the highest powers of $ n $ in $ A $, which give the following:
$$
  A \sim \frac{8zn^2 - z^3}{8n^2} = \frac{8kn^3 - k^3n^3}{8n^2} = \frac{8k - k^3}{8} n
$$
Maximum for $ 8k - k^3 $ is reached for $ k = \frac{2\sqrt{6}}{3} $ giving $ A = \frac{4\sqrt{6}}{9} n $.


\section{Solution}

Let $ Y_i $ is a random variable denoting the number of cycles with length
$ i $. Then $ \E Y_i = \frac{[n]^i}{2i}p^i $.
Let $ Y $ be a random variable denoting the number of cycles with length less than $ g $. Then

\begin{eqnarray*}
  Y &=& \sum_{i=3}^{g-1} Y_i \\
  \E Y &=& \sum_{i=3}^{g-1} \E Y_i \\
    &=& \sum_{i=3}^{g-1} \frac{[n]^i}{2i}p^i \\
    &\le& np^i \\
    &=& n^{(g-1)/g} \\
    &=& o(n) \\
\end{eqnarray*}

We fixed $ p = n^{-1 + 1/g} $ for this to hold.

For $ n $ large enough $ \E Y < \frac{n}{4} $. Then we can use Markov's inequality:
$$
  \P (Y \ge 2 \E Y > 2\frac{n}{4}) \le \frac{1}{2}
$$
If we invert the statement we get: $ \P(Y < \frac{n}{2}) > 1/2\quad (\star)$.

Let $ X_v $ be a \irv denoting weather vertex $ v $ has output degree $ < 2n^{1/(2g)} $.
Let $ X $ be \rv denoting the number of verteces with output degree $ < 2n^{1/(2g)} $.

\begin{eqnarray*}
  X &=& \sum_{v} X_v \\
  \E X &=& \sum_{v} \E X_v \\
    &=& n \P(X_v = 1) \\
    &=& n \P(Z < 2n^{1/(2g)})
\end{eqnarray*}

Where Z is a \rv with binom distribution $ B(n-1, p) $. This means that $ \E
Z = (n-1)p \sim n^{1/g} $, and $ \Var Z = (n-1)p(1 - p) \sim n^{1/g} = \sigma^2
$. Using Chebyshev's inequality for $ Z $ it is easy to see that $ \P(Z
< 2n^{1/(2g)}) = o(1) $ (because $ 2n^{1/(2g)} \sim \sqrt{\E Z} $, so $ Z - \E
Z \sim \E Z = \omega(\sigma) $):
\begin{eqnarray*}
  \P(|Z - \E Z| > t\sigma) &\le& \frac{1}{t^2} \\
  \P(\omega(\sigma) > t\sigma) &\le& \frac{1}{t^2} \\
\end{eqnarray*}

Using this we conclude, that $ \E X = o(n) $, so eventually $ \E
X < \frac{1}{40}n $, then from Markov:
$$
  \P(X \ge 2 \E X > 2 \frac{n}{40}) \le \frac{1}{2}
$$
Inverting the statement yields $ \P(X < \frac{n}{20}) > \frac{1}{2} \quad (\star\star)$.

From $ (\star) $ and $ (\star\star) $ we can conclude that there exists a graph
with $ n $ vertices, which has less than $ n/2 $ cycles with length less than
$ g $ and less than $ \frac{n}{20} $ vertices with degree less than
$ 2n^{1/(2g)} $. We should prove, that removing $ n / 2 $ edges so that the
resulting graph has no cycles with length less than $ g $ will leave at most
$ n / 10 $ vertices with degree $ < n^{1/(2g)} $. This means that at most
$ n / 20 $ of the vertices with degree $ \ge 2n^{1/(2g)} $ should become with
degree $ < n^{1/(2g)} $. So the removed edges should be at least $ (n/20)
n^{1/(2g)} = \omega(n) $, meaning that for $ n $ large enough $ n/2 $ is
less than $ (n/20) n^{1/(2g)} $. Thus there will be at least $ 9n/10 $ vertices
with degree at least $ n^{1/(2g)} $.

% So if we remove at most
% $ n / 2 $ edges to achieve girth $ g $ (by destroying all small cycles), then
% we should check how many vertices with small degree might occur (because we are
% removing edges).  Removing the edges should ``destroy'' more than $ n / 20
% $ vertices with big degree, but we chose them to have degree $ 2n^{1/(2g)} $,
% so to ``destroy'' one such vertex you need to take out at least $ n^{1/(2g)}
% $ edges leading to it. In total $ (n / 20) n^{1/(2g)} = \omega(n) $ edges need
% to be removed. But only $ n / 2 $ are removed to destroy the cycles, so we'll
% have at least $ 9n/10 $ edges with degree at least $ n^{1/(2g)} $.
% the number of vertices with small degree will become at most $ \frac{n}{2.10}
% $, so there will be more than $ \frac{9n'}{10} $ vertices with degree at least
% $ n^{1/2g} $.

% Let $ X_v $ is a random variable, denoting the output degree of vertex $ v $. Obviously, it has binomial distribution $ B(n-1, p) $, which means $ \E X_v = (n-1)p $ and $ \Var X_v = (n-1)p(1 - p) $. We want to use Chebyshev's inequality to prove, that $ \P(X_v < n^{1/(2g)}) = o(1) $.

\section{Solution}
  \subsection*{(a)}

  Fix $ n $.
  \begin{itemize}
    \item \em k = 1 \em\quad $ \binom{n}{1} = n \le  n.e = \bfrac{en}{1}^1 $ \\
    \item \em k \em given \em k - 1 \em \\
    \begin{eqnarray*}
      \binom{n}{k} &=& \binom{n}{k-1} \frac{n-k+1}{k} \\
                   &<& \binom{n}{k-1} \frac{n}{k} \\
                   &\overset{IH}{\le}& \bfrac{en}{k-1}^{k-1} \frac{n}{k} \\
      \bfrac{en}{k}^k &=& \bfrac{en}{k-1}^{k-1} \frac{(k-1)^{k-1}}{k^k}(en) \\
                      &=& \bfrac{en}{k-1}^{k-1} \bfrac{k-1}{k}^{k-1} \frac{en}{k} \\
                      &\ge& \bfrac{en}{k-1}^{k-1} \frac{1}{e} \frac{en}{k} \\
                      &=& \bfrac{en}{k-1}^{k-1} \frac{n}{k}
    \end{eqnarray*}
  \end{itemize}
  \subsection*{(b)}

  We'll calculate $ \underset{n\to\infty}\lim
  \frac{\binom{n}{k}}{\frac{n^k}{k!}} $ and prove it is $ 1 $.

  \begin{eqnarray*}
    \lim_{n\to\infty} \frac{\binom{n}{k}}{\frac{n^k}{k!}}
      &=& \lim_{n\to\infty} \frac{\frac{n!}{(n-k)!k!}}{\frac{n^k}{k!}} \\
      &=& \lim_{n\to\infty} \frac{n!}{(n-k)!n^k} \\
      &\overset{\mathrm{Stirling}}{=}&
          \lim_{n\to\infty} \frac{(1 + o(1)) \sqrt{2\pi n} \bfrac{n}{e}^n}
                                 {(1 + o(1)) \sqrt{2\pi(n-k)} \bfrac{n-k}{e}^{n-k} n^k} \\
      &=& \lim_{n\to\infty} \sqrt{\frac{n}{n-k}} \bfrac{n}{n-k}^{n-k} e^{-k} \\
      &=& \lim_{\substack{n\to\infty\\ \frac{k}{n} \to 0}}
                            \sqrt{\frac{1}{1-\frac{k}{n}}} e^{k+O(\frac{k^2}{n-k})} e^{-k} \\
      &=& \lim_{n\to\infty} e^{O(\frac{k^2}{n-k})} \\
      &\overset{(\star)}{=}& 1
  \end{eqnarray*}
  In $ (\star) $ we use that $ k = o(\sqrt{n}) $ so $ \frac{k^2}{n-k} \le \frac{k^2}{\frac{n}{2}} = o(1) $,
  so it tends to $ 0 $ when $ n \to \infty $.
  
\section{Solution}

Lets fix the number of edges $ m $ and the graph size $ n $ in a graph $ H = (V, E) $.

Lets denote $ \psi = \frac{2n^{3/2}}{3\sqrt{3}\sqrt{m}} $.

Lets choose a random subset $ S \subseteq V $ of the vertices of $ H $, such
that $ |S| = k\psi $ for some positive constant $ k \ge 1 $.

Let $ X_e $ is an \irv showing weather the edge $ e $ is inside the chosen component $ S $. The probability for that is the number of ways to select and edge inside the component divided by all available edges:

$$
  \P(X_e = 1) = \frac{\binom{k\psi}{3}}{\binom{n}{3}} = \frac{[k\psi]^3}{[n]^3} \le \frac{k^3\psi^3}{n^3}
$$

Let $ X $ be an \rv showing the number of edges inside the component $ S $. Thus
\begin{eqnarray*}
  X &=& \sum_e X_e \\
  \E(X) &=& \sum_e \P(X_e = 1) \\
    &=& mk^3\frac{\psi^3}{n^3} \\
    &=& mk^3\frac{8n^{9/2}}{81\sqrt{3}m^{3/2}}\frac{1}{n^3} \\
    &=& k^3 \frac{8n^{3/2}}{81\sqrt{3}\sqrt{m}} \\
    &=& k^3 \frac{4}{27} \psi \\
\end{eqnarray*}

From the \FMM we can choose a component $ S' $, such that $ X \le k^3
\frac{8n^{3/2}}{81\sqrt{3}\sqrt{m}} $. In order to make the component independent, we have to remove at least one vertex per each edge inside $ S' $, and we have to prove that what is left is $ \ge \psi $.

\begin{eqnarray*}
  k\psi - k^3 \frac{4}{27} \psi &\ge& \psi \\
  &\Updownarrow& \\
  (k-1)\psi - k^3 \frac{4}{27} \psi &\ge& 0 \\
  &\Updownarrow& \\
  \underbrace{27k - 27 - 4k^3}_{f(k)} &\ge& 0 \\
\end{eqnarray*}

$$
  f'(k) = 27 - 12k^2 = 12(k - 3/2)(k + 3/2)
$$

And it's easy to check, that $ f(\frac{3}{2}) = 0 $.

So we can choose a component having $ \frac{3}{2} \psi $ vertices, and at most
$ \frac{1}{2} \psi $ edges. Removing a vertex per edge will result in
a component with size at least $ \psi = \frac{2n^{3/2}}{3\sqrt{3}\sqrt{m}} $. \endproof

\section{Solution}

\begin{eqnarray*}
  \E X(X-1) &=& \E (X^2 - X) \\
    &=& \E X^2 - \E X \\
    &=& \E X^2 - (\E X)^2 + (\E X)^2 - \E X \\
    &=& \Var(X) + (\E X)^2 - \E X
\end{eqnarray*}
      
\begin{eqnarray*}
  \E X(X-1) &=& \E (X^2 - X) \\
    &=& \E X^2 - \E X \\
    &=& \E \B{\sum_i Y_i}^2 - \E \B{\sum_i Y_i} \\
    &=& \E \B{\sum_i Y_i^2} + \E \sum_{1\le i,j \le k} Y_iY_j - \E \B{\sum_i Y_i} \\
    &=& \E Z + \E \B{\sum_i \underbrace{Y_i^2 - Y_i}_{=0}} \\
    &=& \E Z
\end{eqnarray*}


\begin{eqnarray*}
  \Var X &=& \E Z - (\E X)^2 + \E X \\
    &=& (1 + o(1) - 1)(\E X)^2 + \E X \\
    &=& o((\E X)^2) + \E X \quad (\star)\\
  \P(X = 0) &<& \P\B{ |X - \E X| \ge \frac{\E X}{2}} \\
    &\overset{\mathrm{Chebyshev}}{\le}&
          4\frac{\Var X}{(\E X)^2} \\
    &\overset{(\star)}{=}& 4\frac{\E X + o((\E X)^2)}{(\E X)^2} \\
    &=& 4\B{ \frac{1}{\E X} + \frac{o((\E X)^2)}{(\E X)^2} } \overset{\E X\to\infty}\longrightarrow 0
\end{eqnarray*}

\section{Solution}

\subsection{A}

Let $ X_{ni} $ be \irv showing weather the $ i^{\mathrm{th}} $ bit is 1. So $ \P(X_{ni} = 1) = p $.

\begin{eqnarray*}
  \E(X_n) &=& \sum_i \P(X_{ni} = 1) \\
  &=& np \\
  \\
  \E Z &=& p^2[n]^2 \\
  \\
  \Var X_n &=& \E Z - (\E X)^2 + \E X \\
           &=& p^2n(n-1) - n^2p^2 + np \\
           &=& np(1 - p) \\
  \\
  \P(X_n < pn - \sqrt{n\log n}) &\le& \P(|X_n - pn| > \sqrt{n\log n}) \\
    &\overset{\mathrm{Chebyshev}}{\le}& \frac{np(1-p)}{(\sqrt{n\log n})^2} \\
    &=& \frac{np(1-p)}{n\log n} \\
    &=& \frac{p(1-p)}{\log n} \\
    &=& O\B{\frac{1}{\log n}} \\
\end{eqnarray*}

\subsection{B}

Let $ Q_{ni} $ is \irv that the sequence $ 101 $ occurs at position $ i $ in the string. Obviously $ \P(Q_{ni}) = p^2(1-p) $. We define

\begin{eqnarray*}
  Q_n &=& \sum_i Q_{ni} \\
  \E Q_n &=& \sum_i^{n-2} \P(Q_{ni} = 1) \\ 
    &=& p^2(1-p)(n-2) \\
  \\
  \E Z &=& \sum_{i,j} \P(Q_{ni}Q_{nj} = 1) \\
     &=& (p^2(1-p))^2 ((n-2)(n-3) - 2(n-4) - 2(n-3)) + p^3(1-p)^2 (2(n-4)) + 0.2(n-3) \\
     &=& (p^2(1-p))^2(n^2 - 9n + 20) + p^3(1-p)^2(2n - 8) \\
  \\
  \Var Q_n &=& \E Z - (\E X)^2 + \E X \\
    &=& (p^2(1-p))^2(-5n + 16) + p^2(1-p)((2p(1-p) + 1)n + o(n)) \\
    &=& \Theta(n)
\end{eqnarray*}

\begin{eqnarray*}
  \P(|Q_n - p^2(1-p)n| > a\sqrt{n}) &\le& \P(|Q_n - \E Q_n| > a\sqrt{n} + O(1)) \\
    &\overset{\mathrm{Chebyshev}}{\le}& \frac{\Var Q_n}{a^2 n + O(\sqrt{n})} \\
    &=& \frac{\Theta(n)}{\Theta(a^2n)} \\
    &=& \Theta\B{\frac{1}{a^2}} \\
  \P(|Q_n - p^2(1-p)n| > a\sqrt{n}) &=& O\B{\frac{1}{a^2}} \\
\end{eqnarray*}
\endproof


\begin{eqnarray*}
 [ \neg ((Q \wedge \neg P) \wedge \neg (Q \wedge R)) \rightarrow (Q
\wedge (Q \rightarrow P) \wedge \neg P)] \wedge (P \vee R) &=& \\
 \lbrack ((Q \wedge \neg P) \wedge \neg (Q \wedge R)) \vee (Q \wedge
(\neg Q \vee P) \wedge \neg P) \rbrack  \wedge (P \vee R) &=& \\
\end{eqnarray*}

\end{document}
