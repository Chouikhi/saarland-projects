\documentclass[a4paper]{article}
\usepackage{ucs}  % unicode
\usepackage[utf8x]{inputenc}
\usepackage{multirow}
\usepackage{array}
% \usepackage[T2A]{fontenc}
% \usepackage[bulgarian]{babel}
\usepackage{graphicx}
% \usepackage{fancyhdr}
% \usepackage{lastpage}
\usepackage{listings}
\usepackage{amsfonts}
\usepackage{amsmath}
% \usepackage{fancyvrb}
% \usepackage[usenames,dvipsnames]{color}
% \setlength{\headheight}{12.51453pt}

%\pagestyle{fancy}
%\fancyhead{}
%\fancyfoot{}

% \cfoot{\thepage\ от \pageref{LastPage}}

% \addto\captionsbulgarian{%
%   \def\abstractname{%
%     Цел на проекта} %\cyr\CYRA\cyrs\cyrt\cyrr\cyra\cyrk\cyrt}}%
% }

% Custom defines:

% TODO remove colorlinks before printing
% \usepackage[unicode,colorlinks]{hyperref}   % this has to be the _last_ command in the preambule, or else - no work
% \hypersetup{urlcolor=blue}
% \hypersetup{citecolor=PineGreen}

\def\d{\mathrm{d}}

\begin{document}

\title{High Level Computer Vision - Exercise 5}
\author{Zornitsa Kostadinova \\ Iskren Ivov Chernev}
\maketitle

\section*{Question 1}
\subsection*{b}
We experimented with the following parameters:
\begin{eqnarray*}
\sigma_1 &=& 5 \\
\sigma_2 &=& 5 \\
C &\in& \{ 0.1, 1, 10^2, 10^4, 10^6 \} \\
\end{eqnarray*}

The margins reported by the SVM were respectively

\includegraphics[scale=0.5]{margin_vs_c.png}

\begin{tabular}{|c|c|}
\hline
C   & margin \\ \hline \hline
$ 0.1 $ & $ 2.790273 $ \\
$ 1 $ & $ 2.134947 $ \\
$ 10^2 $ & $ 1.889752 $ \\
$ 10^4 $ & $ 1.807063 $ \\
$ 10^6 $ & $ 0.061729 $ \\ \hline
\end{tabular}

It is clear that the margin decreases exponentially as the constant $ C $ increases.

\section*{Question 3}
\begin{tabular}{ccc}
        & cell size 8                              & cell size 16 \\
no norm & \includegraphics[scale=0.4]{rpc_b8.png} & \includegraphics[scale=0.4]{rpc_b16.png} \\
norm    & \includegraphics[scale=0.4]{rpc_b8_l2.png} & \includegraphics[scale=0.4]{rpc_b16_l2.png} \\
\end{tabular}

Cell size of 8 gives slightly better results than cell size 16, in both with and without normalization. Also the normalization itself adds between 10 and 15 percent recall.

\subsection*{d}

We need a sliding window on different scales of the image. To make things
faster we can compute the cells (and even blocks) for each scale before the
actual classification step, so the descriptor computations will be reused. The
classification is pretty fast already, because it only does one multiplication
and one addition, so the overall performance will be good.

\end{document}
