\documentclass[a4paper]{article}
\usepackage{ucs}  % unicode
\usepackage[utf8x]{inputenc}
% \usepackage[T2A]{fontenc}
% \usepackage[bulgarian]{babel}
\usepackage{graphicx}
% \usepackage{fancyhdr}
% \usepackage{lastpage}
\usepackage{listings}
\usepackage{amsfonts}
\usepackage{amsmath}
% \usepackage{fancyvrb}
% \usepackage[usenames,dvipsnames]{color}
% \setlength{\headheight}{12.51453pt}

%\pagestyle{fancy}
%\fancyhead{}
%\fancyfoot{}


% TODO remove colorlinks before printing
% \usepackage[unicode,colorlinks]{hyperref}   % this has to be the _last_ command in the preambule, or else - no work
% \hypersetup{urlcolor=blue}
% \hypersetup{citecolor=PineGreen}

\begin{document}

\title{Automata, Games and Verification -- solution to challenge problem 2}
\author{Iskren Ivov Chernev}
\maketitle

\section*{Solution}

Let $ \Sigma = \{ 0, 1 \} $ is a binary alphabet. \\
% Let $ \alpha \in \Sigma^\omega $ is an arbitrary word in that alphabet. \\
% Let $ L \ni \alpha $ is an arbitrary language, containing the word $ \alpha $. \\
Let $ \Sigma^2 = \Sigma \times \Sigma $ is an expansion of the alphabet
$ \Sigma $, in such a way that $ (0, 0), (1, 1) \in \Sigma^2 $ correspond to
$ 0 \in \Sigma $ and $ (0, 1), (1, 0) \in \Sigma^2 $ correspond to $ 1 \in
\Sigma $.  Having this correspondence we can translate any word/language from
$ \Sigma $ in $ \Sigma^2 $ and vice versa. Because each letter in $ \Sigma $ corresponds to
2 letters in $ \Sigma^2 $ then each word $ \alpha \in \Sigma $ is translated to
a set of words $$ \alpha^{\Sigma^2} = \{ \alpha' | (\forall i \in
\omega)(\alpha'(i) \in \{ (0, 0), (1, 1) \} \iff \alpha(i) = 0) \} $$

Let $ \emptyset \ne L'  \subseteq  (\Sigma^2)^\omega $ is a language on the
alphabet $ \Sigma^2 $ translated from the language $ \emptyset \ne L \subseteq
\Sigma^\omega $. We'll prove, that $ pr_1(L') = \Sigma^\omega $.

Let $ \alpha \in L $ is an arbitrary word from $ L $. \\ Let $ \beta \in
\Sigma^\omega $ is also arbitrary chosen word. \\ We'll prove, that $ \beta \in
pr_1(L') $.

Lets construct the word $ \gamma \in \Sigma^\omega $ in the following way:
$$
  \gamma(i) = \alpha(i) \oplus \beta(i)
$$
where $ \oplus $ is addition modulus 2: $ a + b\ (\mathrm{mod}\ 2) $

We'll show that the word $ \alpha'' = (\beta(0), \gamma(0))(\beta(1),
\gamma(1))\dots(\beta(n), \gamma(n))\dots $ is a translation of $ \alpha $.

From the construction of $ \gamma $ it follows, that $ \alpha(i) = \beta(i) \oplus \gamma(i) $ (because $ \oplus $ has this property).

\begin{tabular}{|c||c|c|}
  \hline
  $ \oplus $ & 0 & 1 \\
  \hline \hline
  0 & 0 & 1 \\ \hline
  1 & 1 & 0 \\ \hline
\end{tabular}

This means that $ \alpha'' $ is a translation of $ \alpha $, so $ \alpha'' \in
L' $. Then $ \beta \in pr_1(L') $, and because $ \beta $ was arbitrary chosen,
then $ pr_1(L') = \Sigma^\omega $.

Now the final thing to note is that any language on $ \Sigma^2 $ for which the
letters $ (0, 0), (1, 1) $ and $ (0, 1), (1, 0) $ are interchangeable -- this
means that if we change one letter in one word with its ``buddy'' then the
resulting word is also in the language -- is Buchi recognizable if and only if
the translated language to the alphabet $ \Sigma $ is Buchi recognizable. This
is easy to see if one considers the automata recognizing these languages and
just changes the letters on the edges.

This means that every language on $ \Sigma $ that is not Buchi recognizable,
can be translated to a non Buchi recognizable language on $ \Sigma^2 $, which
in turn has the property that its projection is $ \Sigma^\omega $, which is
Buchi recognizable.

\end{document}
