\documentclass[a4paper]{article}
\usepackage{ucs}  % unicode
\usepackage[utf8x]{inputenc}
% \usepackage[T2A]{fontenc}
% \usepackage[bulgarian]{babel}
\usepackage{graphicx}
% \usepackage{fancyhdr}
% \usepackage{lastpage}
\usepackage{listings}
\usepackage{slashbox}
\usepackage{multirow}
\usepackage{wrapfig}
\usepackage{amsfonts}
\usepackage{amsmath, amsthm, amssymb}
% \usepackage{fancyvrb}
% \usepackage[usenames,dvipsnames]{color}
% \setlength{\headheight}{12.51453pt}

%\pagestyle{fancy}
%\fancyhead{}
%\fancyfoot{}

% \cfoot{\thepage\ от \pageref{LastPage}}

% \addto\captionsbulgarian{%
%   \def\abstractname{%
%     Цел на проекта} %\cyr\CYRA\cyrs\cyrt\cyrr\cyra\cyrk\cyrt}}%
% }

% Custom defines:
\def\definition{Definition:\ }
\def\la{\leftarrow}
\def\vars{\mathrm{Vars}}
\def\occ{\mathrm{Occ}}
\def\lub{\sqcup}
\def\xlub{\underline\lub}
\def\glb{\sqcap}
% \def\dc{bar baz}

% TODO remove colorlinks before printing
% \usepackage[unicode,colorlinks]{hyperref}   % this has to be the _last_ command in the preambule, or else - no work
% \hypersetup{urlcolor=blue}
% \hypersetup{citecolor=PineGreen}

\begin{document}

\newcommand{\aee}[1] {[[#1]]^\sharp}
\newcommand{\cc}[1] {\texttt{#1}}
\newcommand{\s}[1] {\{#1\}}
\newcommand{\e}[1] {[#1]}
% \def\e {[]}
\def\A {\mathcal{A}}
\def\N {\mathcal{N}}
\def\Neg {\mathrm{Neg}}
\def\Pos {\mathrm{Pos}}
\def\Vars {\mathrm{Vars}}
\def\Occ {\mathrm{Occ}}
\def\PP {\mathrm{ProgramPoints}}

\title{Static Program Analysis - Exercise 9}
\author{Iskren Ivov Chernev \\ tutorial group B}

\maketitle

\section{Solution}

\subsection{Must analysis}

\begin{itemize}
  \item Initial state: $ \s{ \e{}, \e{}, \e{}, \e{} } $.
  \item Entry to 1: $ \s{ \e{a}, \e{}, \e{}, \e{} } $ (this gets combined with
        $ \bot $, I won't state this again).
  \item 1 to 2: $ \s{ \e{b}, \e{a}, \e{}, \e{} } $.
  \item 2 to 3: $ \s{ \e{a}, \e{b}, \e{}, \e{} } $.
  \item 3 to 4: $ \s{ \e{c}, \e{a}, \e{b}, \e{} } $.
  \item 4 to 1: We combine in a conservative way: $ \s{ \e{}, \e{a}, \e{}, \e{}
        } $.
  \item 1 to 2: $ \s{ \e{b}, \e{}, \e{a}, \e{} } $, combining with previous
        gives $ \s{ \e{b}, \e{}, \e{a}, \e{} } $.
  \item 2 to 3: $ \s{ \e{a}, \e{b}, \e{}, \e{} } $, this is no different than
        previous one, so continue with 4 to 5.
  \item 4 to 5: $ \s{ \e{d}, \e{c}, \e{a}, \e{b} } $.
  \item 5 to 6: $ \s{ \e{e}, \e{d}, \e{c}, \e{a} } $.
  \item 6 to Exit: $ \s{ \e{b}, \e{e}, \e{d}, \e{c} } $.

\end{itemize}

\subsection{May analysis}

\begin{itemize}
  \item Initial state: $ \s{ \e{}, \e{}, \e{}, \e{} } $.
  \item Entry to 1: $ \s{ \e{a}, \e{}, \e{}, \e{} } $.
  \item 1 to 2: $ \s{ \e{b}, \e{a}, \e{}, \e{} } $.
  \item 2 to 3: $ \s{ \e{a}, \e{b}, \e{}, \e{} } $.
  \item 3 to 4: $ \s{ \e{c}, \e{a}, \e{b}, \e{} } $ (up to now no difference than must analysis).
  \item 4 to 1: Combine: $ \s{ \e{c, a}, \e{}, \e{b}, \e{} } $.
  \item 1 to 2: $ \s{ \e{b}, \e{c, a}, \e{}, \e{} } $, combining with previous gives same (new) result.
  \item 2 to 3: $ \s{ \e{a}, \e{b}, \e{c}, \e{} } $, combining with previos gives same (new) result.
  \item 3 to 4: $ \s{ \e{c}, \e{a}, \e{b}, \e{} } $, this is the same as previous value.
  \item from now on the analysis is the same as the may analysis
  \item 4 to 5: $ \s{ \e{d}, \e{c}, \e{a}, \e{b} } $.
  \item 5 to 6: $ \s{ \e{e}, \e{d}, \e{c}, \e{a} } $.
  \item 6 to Exit: $ \s{ \e{b}, \e{e}, \e{d}, \e{c} } $.
\end{itemize}

\end{document}
