\documentclass[a4paper]{article}
\usepackage{ucs}  % unicode
\usepackage[utf8x]{inputenc}
% \usepackage[T2A]{fontenc}
% \usepackage[bulgarian]{babel}
\usepackage{graphicx}
% \usepackage{fancyhdr}
% \usepackage{lastpage}
\usepackage{listings}
\usepackage{slashbox}
\usepackage{multirow}
\usepackage{wrapfig}
\usepackage{amsfonts}
\usepackage{amsmath, amsthm, amssymb}
% \usepackage{fancyvrb}
% \usepackage[usenames,dvipsnames]{color}
% \setlength{\headheight}{12.51453pt}

%\pagestyle{fancy}
%\fancyhead{}
%\fancyfoot{}

% \cfoot{\thepage\ от \pageref{LastPage}}

% \addto\captionsbulgarian{%
%   \def\abstractname{%
%     Цел на проекта} %\cyr\CYRA\cyrs\cyrt\cyrr\cyra\cyrk\cyrt}}%
% }

% Custom defines:
\def\definition{Definition:\ }
\def\la{\leftarrow}
\def\vars{\mathrm{Vars}}
\def\occ{\mathrm{Occ}}
% \def\dc{bar baz}

% TODO remove colorlinks before printing
% \usepackage[unicode,colorlinks]{hyperref}   % this has to be the _last_ command in the preambule, or else - no work
% \hypersetup{urlcolor=blue}
% \hypersetup{citecolor=PineGreen}

\begin{document}

\newcommand{\aee}[1] {[[#1]]^\sharp}
\newcommand{\cc}[1] {\texttt{#1}}
\def\A {\mathcal{A}}
\def\N {\mathcal{N}}
\def\NonZero {\mathrm{NonZero}}
\def\Zero {\mathrm{Zero}}
\def\Vars {\mathrm{Vars}}
\def\Occ {\mathrm{Occ}}

\title{Static Program Analysis - Exercise 4}
\author{Iskren Ivov Chernev \\ tutorial group B}

\maketitle

\section{Not yet}
\section{Solution}

The basis of the carrier will be the flat lattice $ \mathbb{S} = \{ +, -, 0,
\top \} $, where $ \top $ is the top, and the other elements are not
comparable. Then defining the carrier is trivial $$ \mathbb{D} = (\Vars \to
\mathbb{S})_{\bot} $$

It is a complete lattice with $ \sqcup $ and $ \sqcap $ defined accordingly.

The description relation $ \Delta $ is defined by $$ x \Delta z \quad \iff
\quad (0 < x \land z = +) \lor (0 = x \land z = +) \lor (x < 0 \land z = -)
\lor z = \top $$

The concretization $ \gamma $ follows easily
$$
\gamma a = \begin{cases}
  [1, +\infty)        & a = + \\
  0                   & a = 0 \\
  (-\infty, -1]       & a = - \\
  (-\infty, +\infty)  & a = \top
\end{cases}
$$

Now we should define the abstract operator effects

$$
  x +^\sharp y = \begin{cases}
    \top      & \text{if $ x = \top $ or $ y = \top $}\\
    x         & \text{if $ y = 0 $} \\
    y         & \text{if $ x = 0 $} \\
    x         & \text{if $ x = y $} \\
    \top      & \text{otherwise}
  \end{cases}
$$

$$
  -^\sharp x = \begin{cases}
    \top    & \text{if $ x = \top $} \\
    -       & \text{if $ x = + $} \\
    +       & \text{if $ x = - $} \\
    0       & \text{if $ x = 0 $} \\
  \end{cases}
$$

$$
  x \cdot^\sharp y = \begin{cases}
    \top      & \text{if $ x = \top \land y \ne 0 $ or $ x \ne 0 \land y = \top $} \\
    +         & \text{if $ (x, y) \in \{ (+, +), (-, -) \} $} \\
    -         & \text{if $ (x, y) \in \{ (+, -), (-, +) \} $} \\
    0         & \text{if $ x = 0 $ or $ y = 0 $ } \\
  \end{cases}
$$

$$
  x /^\sharp y = \begin{cases}
    +         & \text{if $ (x, y) \in \{ (+, +), (-, -) \} $} \\
    -         & \text{if $ (x, y) \in \{ (+, -), (-, +) \} $} \\
    0         & \text{if $ x = 0 $ and $ \lnot(y \sqsupseteq 0) $ } \\
    \top      & \text{otherwise} \\
  \end{cases}
$$

We shall define the boolean operators abstract effects. Here we are using the
flat lattice $ \{ false, true, \top \} $.

$$
  x <^\sharp y = \begin{cases}
    true    & \text{if $ (x, y) \in \{ (-, 0), (0, +), (-, +) \} $} \\
    false   & \text{if $ (x, y) \in \{ (0, -), (+, 0), (+, -), (0, 0) \} $} \\
    \top    & \text{otherwise} \\
  \end{cases}
$$

$$
  x \le^\sharp y = \begin{cases}
    true    & \text{if $ (x, y) \in \{ (-, 0), (0, +), (-, +), (0, 0) \} $} \\
    false   & \text{if $ (x, y) \in \{ (0, -), (+, 0), (+, -) \} $} \\
    \top    & \text{otherwise} \\
  \end{cases}
$$

$$
  x =^\sharp y = \begin{cases}
    true    & \text{if $ x = 0 $ and $ y = 0 $} \\
    false   & \text{if $ (x, y) \in \{ (a, b)\ \in \mathbb{S}^2 |\ a \ne \top \land b \ne \top \land a \ne b \} $} \\
    \top    & \text{otherwise} \\
  \end{cases}
$$

The other boolean operators are analogous.

Now the definition of the abstract edge effects

\begin{eqnarray*}
  \aee{;} D &=& D \\
  \aee{M[e_1] = e_2} D &=& D \\
  \aee{\Zero(e)} D &=& \begin{cases}
    D     & \text{if $ \aee{e} D \sqsupseteq false $} \\
    \bot  & \text{otherwise} \\
  \end{cases} \\
  \aee{\NonZero(e)} D &=& \begin{cases}
    \bot  & \text{if $ \aee{e} D = false $} \\
    D     & \text{otherwise} \\
  \end{cases} \\
  \aee{x \leftarrow e} &=& D \oplus \{ x \leftarrow \aee{e} D \} \\
  \aee{x \leftarrow M[e]\ } &=& D \oplus \{ x \leftarrow \top \} \\
\end{eqnarray*}

This is a forward analysis, so the abstract path effect $ \aee{\pi} $ for path $ \pi = k_1 k_2 \dots k_n $ is defined as $$ \aee{\pi} = \aee{k_n} \circ \aee{k_{n-1}} \circ \dots \circ \aee{k_1} $$

The MOP solution is just the least upper bound over all paths, with initial value $ \top = \{ x \to \top\ |\ x \in \Vars \} $
$$
  \mathcal{A}^\star[v] = \bigsqcup \{ \aee{\pi} \top\ |\ \pi : start \to^\star v \}
$$
Where of course $ \sqcup $ is defined as usual for the complete lattice $ \mathbb{D} $.

The system of inequalities is defined by $ A_{start} \sqsupseteq \top $ and for all edges $ k = (u, lab, v) $
$$
  A_v \sqsupseteq \aee{lab} A_u \quad \exists \text{edge } k = (u, lab, v)
$$

Source code:

\begin{verbatim}
TYPE

  AbstractVal = flat(snum)           /* -1 0 1 and top */
  AbstractBind = Var -> AbstractVal
  AbstractBindL = lift(AbstractBind) /* use only (x -> {-1, 0, 1, top}) + bot */

PROBLEM Sign_propagation

  direction  : forward
  carrier    : AbstractBindL
  init       : bot
  init_start : abTop()
  combine    : lub


TRANSFER

  ASSIGN(variable, expression) =
    let f <= @ in
      lift(f \ [ variable -> evalAB(expression, f) ])

  IF(expression),    true_edge  = let f <= @ in posEE(expression, f)
  IF(expression),    false_edge = let f <= @ in negEE(expression, f)

  WHILE(expression), true_edge  = let f <= @ in posEE(expression, f)
  WHILE(expression), false_edge = let f <= @ in negEE(expression, f)

  CALL(_, param, expression), call_edge = 
    let f <= @ in
      lift(f \ [ param -> evalAB(expression, f) ])

  CALL(_, _, _), local_edge = bot

  END(_, param) =
    let f <= @ in
      lift(f \ [ param -> top ])

SUPPORT

  abTop :: -> AbstractBindL
  abTop() = lift([-> top])

  posEE :: Expression * AbstractBind -> AbstractBindL
  posEE(expression, bind) =
    let val = bEvalAB(expression, bind) in
    if  val = lift(0) then bot else lift(bind) endif

  negEE :: Expression * AbstractBind -> AbstractBindL
  negEE(expression, bind) =
    let val = bEvalAB(expression, bind) in
    if  val = lift(1) then bot else lift(bind) endif


  evalAB :: Expression * AbstractBind -> AbstractVal
  evalAB(expression, bind) =
    case expType(expression) of
      "ARITH_BINARY" =>
        let x = evalAB(expSubLeft(expression), bind),
            y = evalAB(expSubRight(expression), bind) in
        case expOp(expression) of
          "+" => if x = top || y = top    then top
            else if y = lift(0)           then x
            else if x = lift(0)           then y
            else if x = y                 then x
            else                               top
            endif endif endif endif;
          "-" => if x = top || y = top    then top
            else if y = lift(0)           then x
            else if x = lift(0)           then lift(-drop(y))
            else if drop(x) = -drop(y)    then x
            else                               top
            endif endif endif endif;
          "*" => if x = lift(0) || y = lift(0) then lift(0)
            else if x = top     || y = top     then top
            else                                    lift(drop(x) * drop(y))
            endif endif;
          "/" => if x = top || y = top || y = lift(0) then top
            else if x = lift(0)                       then lift(0)
            else                                      lift(drop(x) * drop(y))
            endif endif;
        endcase;
      "ARITH_UNARY" => let x <= evalAB(expSub(expression), bind) in lift(-x);
      "VAR" => bind(expVar(expression));
      "CONST" => lift(sgn(expVal(expression)));
      _ => error("incorrect expression type passed to evalAB");
    endcase

  bEvalAB :: Expression * AbstractBind -> AbstractVal
  bEvalAB(expression, bind) =
    case expType(expression) of
      "BOOL_BINARY" =>
        let x = evalAB(expSubLeft(expression), bind),
            y = evalAB(expSubRight(expression), bind) in
        case expOp(expression) of
          "="  => bEvalEq(x, y);
          "<>" => bNot(bEvalEq(x, y));
          "<"  => bEvalLt(x, y);
          ">"  => bEvalLt(y, x);
          "<=" => bNot(bEvalLt(y, x));
          ">=" => bNot(bEvalLt(x, y));
        endcase;
      "BOOL_UNARY" =>
        let x = bEvalAB(expSub(expression), bind) in
        bNot(x);
      _ => error("incorrect expression type passed to bEvalAB");
    endcase

  bEvalEq :: AbstractVal * AbstractVal -> AbstractVal
  bEvalEq(xl, yl) = let x <= xl, y <= yl in
         if x = 0 && y = 0 then lift(1)
    else if x != y         then lift(0)
    else                        top
    endif endif

  bEvalLt :: AbstractVal * AbstractVal -> AbstractVal
  bEvalLt(xl, yl) = let x <= xl, y <= yl in
         if x < y          then lift(1)
    else if x > y || x = 0 then lift(0)
    else                        top
    endif endif

  bNot :: AbstractVal -> AbstractVal
  bNot(xl) = let x <= xl in lift(1 - x)

  subExpressions :: Expression -> ExpressionSet
  subExpressions(expression) =
    case expType(expression) of
      "ARITH_BINARY" => subExpressions(expSubLeft(expression)) lub
                        subExpressions(expSubRight(expression));
      "ARITH_UNARY"  => subExpressions(expSub(expression));
      "BOOL_BINARY"  => subExpressions(expSubLeft(expression)) lub
                        subExpressions(expSubRight(expression));
      "BOOL_UNARY"   => subExpressions(expSub(expression));
      _              => {}; 
    endcase
    + expression

  variables :: Expression -> VarSet
  variables(expression) =
    { expVar(exp) | exp in subExpressions(expression);
                    expType(exp) = "VAR" }
\end{verbatim}

\end{document}
