\documentclass[a4paper]{article}
\usepackage{ucs}  % unicode
\usepackage[utf8x]{inputenc}
% \usepackage[T2A]{fontenc}
% \usepackage[bulgarian]{babel}
\usepackage{graphicx}
% \usepackage{fancyhdr}
% \usepackage{lastpage}
\usepackage{listings}
\usepackage{amsfonts}
\usepackage{amsmath}
% \usepackage{fancyvrb}
% \usepackage[usenames,dvipsnames]{color}
% \setlength{\headheight}{12.51453pt}

%\pagestyle{fancy}
%\fancyhead{}
%\fancyfoot{}

% \cfoot{\thepage\ от \pageref{LastPage}}

% \addto\captionsbulgarian{%
%   \def\abstractname{%
%     Цел на проекта} %\cyr\CYRA\cyrs\cyrt\cyrr\cyra\cyrk\cyrt}}%
% }

% Custom defines:

% TODO remove colorlinks before printing
% \usepackage[unicode,colorlinks]{hyperref}   % this has to be the _last_ command in the preambule, or else - no work
% \hypersetup{urlcolor=blue}
% \hypersetup{citecolor=PineGreen}

\def\d{\mathrm{d}}
\def\T{\mathrm{T}}
\def\R{\mathbb{R}}
\def\N{\mathbb{N}}
\def\LDR{Levinson-Durbin recursion\;}
\newcommand{\TT}[1] {\texttt{#1}}
\newcommand{\norm}[1] {\left\lVert #1 \right\rVert}
\newcommand{\B}[1] {\left(#1\right)}

\begin{document}

\title{Digital Signal Processing - Exercise 8}
\author{Iskren Ivov Chernev}
\maketitle

\section*{Exercise 1}

\subsection*{1.1}

I couldn't proof that the matrix is positive-semidefinite.

Let $ \phi $ is an eigenvector with eigenvalue $ \lambda $. So $ R\phi
= \lambda\phi $. Multiply this equation by $ \phi^\T $ on the left side to get
$ 0 \le \phi^\T R \phi = \phi^\T \lambda \phi = \lambda \phi^\T \phi = \lambda
\norm{\phi}^2 $. But $ \norm{\phi}^2 \ge 0 $, so $ \lambda \ge 0 $. Also,
because $ R $ is symetric ($ R_{ij} = R_{ji} $), $ \lambda^\T R\lambda \in \R
$, $ \norm{\phi}^2 \in \R $, so $ \lambda \in \R $ Since $ \lambda $ was an
arbitrary eigenvalue, we proved that all eigenvalues are non-negative and real.

\subsection*{1.2}

In order to proof that R has Toeplitz structure we need to proof that $ R_{ij}
= R_{i+d,j+d} = R_{|j-i|} $. Lets calculate $ R_{i+d,j+d} - R_{ij} $, where
$ i,j \in \{ 0, \dots, n \} $ and $ d \in \N $, such that $ i + d, j + d \in \{
0, \dots, n \} $.

\begin{eqnarray*}
R_{i+d,j+d} - R_{ij}
  &=& \frac{1}{n+1}\B{ \sum_{k=0}^{n} x[k+i+d]x[k+j+d] - \sum_{k=0}^{n} x[k+i]x[k+j] } \\
  &=& \frac{1}{n+1}\B{ \sum_{k=d}^{n+d} x[k+i]x[k+j] - \sum_{k=0}^{n} x[k+i]x[k+j] } \\
  &=& \frac{1}{n+1}\B{ \sum_{k=n+1}^{n+d} x[\underbrace{k+i}_{\in [n+1,2n]}]x[\underbrace{k+j}_{\in [n+1,2n]}]
                     - \sum_{k=0}^{d-1} x[\underbrace{k+i}_{\in [0,n-1]}]x[\underbrace{k+j}_{\in [0,n-1]}] } \\
\end{eqnarray*}

Now, if the signal is cyclic: $ x[i] = x[i+(n+1)k] $ for any $ i, k \in \N $,
then $ R_{i+d,j+d} = R_{ij} $. It is clear, that $ R_{d} = R_{-d} = R_{n+1+d} $,
so R has the desired structure.

\[
\begin{pmatrix}
  R_{0} & R_{1} & \cdots & R_{n} \\
  R_{n+2} & \ddots & \ddots & \vdots \\
  \vdots & \ddots & \ddots & R_{1} \\
  R_{2n+1} & \cdots & R_{n+2} & R_{0} \\
\end{pmatrix}
\]

NOTE: This is slightly different than the formula given on the exercise, but I double checked my version and I think it's correct :)

\section*{Exercise 2}

\subsection*{2.1}

\LDR solves the following problem:

\[
\begin{pmatrix}
  r_{0} & r_{1} & \cdots & r_{p-1} \\
  r_{1} & \ddots & \ddots & \vdots \\
  \vdots & \ddots & \ddots & r_{1} \\
  r_{p-1} & \cdots & r_{1} & r_{0} \\
\end{pmatrix} \begin{pmatrix}
  a_1^p \\ a_2^p \\ \vdots \\ a_p^p
\end{pmatrix} = \begin{pmatrix}
  r_1 \\ r_2 \\ \vdots \\ r_p
\end{pmatrix}
\]

recursively. That is, it finds the solution of the above equation given the
solution of:

\[
\begin{pmatrix}
  r_{0} & r_{1} & \cdots & r_{p-2} \\
  r_{1} & \ddots & \ddots & \vdots \\
  \vdots & \ddots & \ddots & r_{1} \\
  r_{p-2} & \cdots & r_{1} & r_{0} \\
\end{pmatrix} \begin{pmatrix}
  a_1^{p-1} \\ a_2^{p-1} \\ \vdots \\ a_{p-1}^{p-1}
\end{pmatrix} = \begin{pmatrix}
  r_1 \\ r_2 \\ \vdots \\ r_{p-1}
\end{pmatrix}
\]

The runtime is $ O(p^2) $, because each recursive step needs $ O(p)
$ operations to compute the solution given the smaller solution.

The basic idea is:

\[
\begin{pmatrix} -1 \\ a_1^p \\ \vdots \\ a_{p-1}^p \\ a_p^p \end{pmatrix} =
\begin{pmatrix} -1 \\ a_1^{p-1} \\ \vdots \\ a_{p-1}^{p-1} \\ 0 \end{pmatrix} - k_p
\begin{pmatrix} 0 \\ a_{p-1}^{p-1} \\ \vdots \\ a_1^{p-1} \\ -1 \end{pmatrix}
\]

\subsection*{2.2}

\begin{eqnarray*}
q^{P-1} &=& -r_{P} + \sum_{k=1}^{P-1} r_{P-k}a_k^{P-1} \\
k_{P} &=& - q^{P-1} / E^{P-1} \\
a_P^P &=& k_{P} \\
a_i^P &=& a_{i}^{P-1} - k_p a_{P-i}^{P-1} \\
E^{P} &=& E^{P-1} - [q^{P-1}]^2 / E^{P-1} \\
\end{eqnarray*}

\subsection*{2.3}

Implementation in \TT{ldr}.

\subsection*{2.4}

Implementation in \TT{test\_ldr}.

\end{document}
