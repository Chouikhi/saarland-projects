\documentclass[a4paper]{article}
\usepackage{ucs}  % unicode
\usepackage[utf8x]{inputenc}
% \usepackage[T2A]{fontenc}
% \usepackage[bulgarian]{babel}
\usepackage{graphicx}
% \usepackage{fancyhdr}
% \usepackage{lastpage}
\usepackage{listings}
\usepackage{amsfonts}
\usepackage{amsmath}
% \usepackage{fancyvrb}
% \usepackage[usenames,dvipsnames]{color}
% \setlength{\headheight}{12.51453pt}

%\pagestyle{fancy}
%\fancyhead{}
%\fancyfoot{}

% \cfoot{\thepage\ от \pageref{LastPage}}

% \addto\captionsbulgarian{%
%   \def\abstractname{%
%     Цел на проекта} %\cyr\CYRA\cyrs\cyrt\cyrr\cyra\cyrk\cyrt}}%
% }

% Custom defines:

% TODO remove colorlinks before printing
% \usepackage[unicode,colorlinks]{hyperref}   % this has to be the _last_ command in the preambule, or else - no work
% \hypersetup{urlcolor=blue}
% \hypersetup{citecolor=PineGreen}

\def\d{\mathrm{d}}
\def\T{\mathrm{T}}
\def\R{\mathbb{R}}
\def\N{\mathbb{N}}
\def\LDR{Levinson-Durbin recursion\;}
\newcommand{\TT}[1] {\texttt{#1}}
\newcommand{\norm}[1] {\left\lVert #1 \right\rVert}
\newcommand{\B}[1] {\left(#1\right)}
\def\F {\mathcal{F}}
\def\w {\omega}

\begin{document}

\title{Digital Signal Processing - Exercise 8}
\author{Iskren Ivov Chernev}
\maketitle

\section*{Exercise 1}

\subsection*{1.1}

The error in time domain is given by the formula:

\[
  e(n) = s(n) - h(n) * x(n)
\]

Converting this equation to frequency domain gives:

\[
  J(\w_k) = S(\w_k) - H(\w_k) X(\w_k)
\]

\subsection*{1.2}

We'll take the derivative of $ E[J(\w_k)] $ by $ H(\w_k) $, and make it equal to 0.

\begin{eqnarray*}
\frac{\d E[J^2(\w_k)]}{\d H(\w_k)} &=& 0 \\
0 &=& E\left[ \frac{\d (S(\w_k) - H(\w_k) X(\w_k))^2}{ \d H(\w_k) } \right] \\
0 &=& E[ 2(S(\w_k) - H(\w_k)X(\w_k))X(\w_k) ] \\
0 &=& E[ S(\w_k)X(\w_k) ] - H(\w_k) E [ X(\w_k)X(\w_k) ] \\
H(\w_k) &=& \frac{E[ S(\w_k)X(\w_k) ]}{E [ X(\w_k)X(\w_k) ]} \\
H(\w_k) &=& \frac{\phi_{sx}(\w_k)}{\phi_{xx}(\w_k)} \\
\end{eqnarray*}

\subsection*{1.3}

If the signal and the noise are uncorrelated, then we can use $ X(\w_k) = S(\w_k) + N(\w_k) $ to get:

\begin{eqnarray*}
H(\w_k)
  &=& \frac{E[ S(\w_k)X(\w_k) ]}{E [ X(\w_k)X(\w_k) ]} \\
  &=& \frac{E[ S(\w_k)(S(\w_k) + N(\w_k)) ]}{E [ (S(\w_k) + N(\w_k))^2 ]} \\
  &=& \frac{E[ S(\w_k)S(\w_k) + S(\w_k)N(\w_k)) ]}{E [ S(\w_k)S(\w_k) + 2S(\w_k)N(\w_k) + N(\w_k)N(\w_k) ]} \\
  &=& \frac{E[ S(\w_k)S(\w_k) ] + \overbrace{E[ S(\w_k)N(\w_k)) ]}^{0}}
           {E[ S(\w_k)S(\w_k) ] + 2\underbrace{E[ S(\w_k)N(\w_k) ]}_{0} + E[ N(\w_k)N(\w_k) ]} \\
  &=& \frac{\Phi_{ss}(\w_k)}{\Phi_{ss}(\w_k) + \Phi_{nn}(\w_k)} \\
\end{eqnarray*}

\subsection*{1.4}

If we have $ H(\w_k) $ we can obtain $ \hat S(\w_k) = H(\w_k)X(\w_k) $, and then using reverse Fourier transform we obtain the clean signal $ \hat s(n) = \F^{-1}(\hat S(\w_k)) $.

\subsection*{1.5}

If we assume that the noise and the signal are uncorrelated then we get
\[
  H(\w_k) = \B{1 + SNR_k^{-1}}^{-1}
\]

\subsection*{1.6}

Implementation in script \TT{plot\_amplitude}. The greater the signal to noise
ratio, the closer the frequency response is to 1 ($ \log = 0 $). This means
that the smaller the noise compared to the signal, the less ``work'' needs to
be done by the filter.

\end{document}
