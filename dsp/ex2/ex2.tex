\documentclass[a4paper]{article}
\usepackage{ucs}  % unicode
\usepackage[utf8x]{inputenc}
% \usepackage[T2A]{fontenc}
% \usepackage[bulgarian]{babel}
\usepackage{graphicx}
% \usepackage{fancyhdr}
% \usepackage{lastpage}
\usepackage{listings}
\usepackage{amsfonts}
\usepackage{amsmath}
% \usepackage{fancyvrb}
% \usepackage[usenames,dvipsnames]{color}
% \setlength{\headheight}{12.51453pt}

%\pagestyle{fancy}
%\fancyhead{}
%\fancyfoot{}

% \cfoot{\thepage\ от \pageref{LastPage}}

% \addto\captionsbulgarian{%
%   \def\abstractname{%
%     Цел на проекта} %\cyr\CYRA\cyrs\cyrt\cyrr\cyra\cyrk\cyrt}}%
% }

% Custom defines:
\def\rbc{foo bar}
\def\dc{bar baz}

% TODO remove colorlinks before printing
% \usepackage[unicode,colorlinks]{hyperref}   % this has to be the _last_ command in the preambule, or else - no work
% \hypersetup{urlcolor=blue}
% \hypersetup{citecolor=PineGreen}

\begin{document}

\title{Digital Signal Processing - Exercise 2}
\author{Iskren Ivov Chernev}
\maketitle

\def\sm{\mathrm{sm}}
\def\P{\mathbb{P}}
\def\E{\mathbb{E}}
\def\irv{i.r.v.\;}
\def\rv{r.v.\;}
\def\FMM{F.M.M.\;}
\def\endproof{$\square$}
\def\Var{\mathrm{Var}}
\def\Cov{\mathrm{Cov}}
\def\CS{\mathrm{CS}}
\def\CCS{\mathrm{CCS}}
\def\R{\mathrm{R}}
\newcommand{\FL}[1] {\left\lfloor #1 \right\rfloor}
% \newcommand{\irv}{i.r.v\;}
\newcommand{\bfrac}[2] {\left(\frac{#1}{#2}\right)}
\newcommand{\B}[1] {\left(#1\right)}
% \newcommand{\Binom}[2] {\B{

\section*{Exercise 1}

\paragraph*{Subtask 1.1}
Lets consider a kernel with size $ K \in \R^{h\times{}w} $ and an image $ I \in
\R^{n\times{}m} $. Lets consider the pixel $ x, y $ in the convolved image:
\begin{eqnarray*}
  I_2(x, y) = \sum_{i = -\lfloor \frac{h}{2} \rfloor}^{\lfloor \frac{h-1}{2} \rfloor}
              \sum_{j = -\lfloor \frac{w}{2} \rfloor}^{\lfloor \frac{w-1}{2} \rfloor}
                K(i, j) I(x + i, y + j)
\end{eqnarray*}
From this formula it is clear that we access the image off its bounds by
$ \lfloor \frac{h}{2} \rfloor, \lfloor \frac{w-1}{2} \rfloor, \lfloor
\frac{h-1}{2} \rfloor, \lfloor \frac{w}{2} \rfloor $ respectively in directions
up, left, down, right. So we need the strip size $ s $ to be equal to the maximum of these numbers, that is
$$
  s = \max\B{\FL{\frac{h}{2}}, \FL{\frac{w-1}{2}}, \FL{\frac{h-1}{2}}, \FL{\frac{w}{2}}}
$$
Of course, we may mirror the image as much as it needs in each direction, but
that is not necessary.

The function \texttt{imgmirror} does the mirroring, given an image and a strip size.

\subsection*{Subtask 1.2}

The function \texttt{boxfilter} does the actual filtering. It calculates the strip size as described previously, mirrors the image and runs a box filter on top. The actual convolution is performed in the private function \texttt{i\_conv2}.

The runtime of the algorithm is clearly $ \Theta(n.m.h.w) $ where the image has
size $ n \times m $ and the filter has size $ h \times w $. That is because for
every pixel in the image we have to consider $ h \times w $ neighbouring pixels
and multiply them by the appropriate filter value and sum them all up.

\subsection*{Subtask 1.3}

The function \texttt{smooth\_verifier} can be used to test the performance of
the box filter implemented in the previous subtask. It receives the parameters
for salt-pepper intensity and the box filter size. It then displays the
original image, the noisy image and the filtered image.

\section*{Exercise 2}

\paragraph*{Subtask 2.1}

The function \texttt{histogram} implements the desired functionality.
Computations of bins is done slightly differently: for each pixel with range
$ [0, 255] $ is divided by $ 256 $ to get a real number in $ [0, 1) $, and then
multiplied by \texttt{num\_bins}, to get a number in $ [0, \mathrm{num\_bins}) $.
Rounding the result down produces a number in $ \{ 0, 1, \dots,
\mathrm{num\_bins}-1 \} $, which has exactly \texttt{num\_bins} possible values.

\paragraph*{Subtask 2.2}

The function \texttt{segmented\_histogram} implements the algorithm that
computes histograms of multiple parts (segments) of an image for all channels
(rgb) and concatenates the histograms together.

The function \texttt{ln\_dist} computes the n-norm distance (Manhattan,
euclidean etc). Pass $ n = -1 $ to get infinity-norm.

The function \texttt{distance\_test} computes the histograms and distances
between them for images \texttt{img1.jpg, img2.jpg, img3.jpg, img4.jpg}. It
then displays the distances between the histograms.

\end{document}
