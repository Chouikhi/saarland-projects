\documentclass[a4paper]{article}
\usepackage{ucs}  % unicode
\usepackage[utf8x]{inputenc}
% \usepackage[T2A]{fontenc}
% \usepackage[bulgarian]{babel}
\usepackage{graphicx}
% \usepackage{fancyhdr}
% \usepackage{lastpage}
\usepackage{listings}
\usepackage{amsfonts}
\usepackage{amsmath}
% \usepackage{fancyvrb}
% \usepackage[usenames,dvipsnames]{color}
% \setlength{\headheight}{12.51453pt}

%\pagestyle{fancy}
%\fancyhead{}
%\fancyfoot{}

% \cfoot{\thepage\ от \pageref{LastPage}}

% \addto\captionsbulgarian{%
%   \def\abstractname{%
%     Цел на проекта} %\cyr\CYRA\cyrs\cyrt\cyrr\cyra\cyrk\cyrt}}%
% }

% Custom defines:

% TODO remove colorlinks before printing
% \usepackage[unicode,colorlinks]{hyperref}   % this has to be the _last_ command in the preambule, or else - no work
% \hypersetup{urlcolor=blue}
% \hypersetup{citecolor=PineGreen}

\def\d{\mathrm{d}}

\begin{document}

\title{Digital Signal Processing - Exercise 4}
\author{Iskren Ivov Chernev}
\maketitle

% \newcommand{\Binom}[2] {\B{

\section*{Exercise 1}

\subsection*{1.1}

Implemented in \texttt{pre\_emph}. This is a simple linear filter, using only
the original sequence (not the filtered one). This is a high-pass filter,
because it increases the magnitude of the higher frequencies, and decreases the
magnitude of the lower frequencies.

\subsection*{1.2}

The windowing is implemented in \texttt{windowing}. The number of frames for
given shift and width are calculated in the function
\texttt{get\_frame\_count}. The formula is $ \mathrm{frameCount}
= (\mathrm{len} - \mathrm{width}) / \mathrm{shift} $. This is basically how
many shifts we can do in the signal, but first discarding the ``final'' width
(on the last frame the total shift + width must be within the signal).

\subsection*{1.3}

The spectogram generation is implemented in \texttt{spectogram}, the testing
function is \texttt{show\_spectogram}. It is clear that in the moments of
silence there is a constant ``background'' frequency, and in the moments of
sound the frequency distribution is different depending on the actual sounds.

\subsection*{1.4}

Implemented in \texttt{show\_mel}. The filter banks are plotted in L different
graphs.

\subsection*{1.5}

The MFCC algorithm is implemented in \texttt{mfcc}, the testing function is
\texttt{show\_mfcc}. To see both the MFCC and the spectrogram use
\texttt{show\_all}. The similarities are big -- the MFCC has just L (24)
frequency bins, while the spectrogram has 256. The spectrogram is more noisy and
without the logarithm fixation of the magnitudes of the frequencies.

\section*{Exercise 2}

\subsection*{2.1}

\begin{eqnarray*}
\epsilon(a,b) &=& \sum_{i = 1}^{N} (y_i - ax_i - b)^2 \\
\frac{\d\epsilon(a, b)}{\d b} &=& 0 \\
0 &=& \sum_{i = 1}^{N} 2(y_i - ax_i - b)(-1) \\
0 &=& \sum_{i = 1}^{N} y_i - a\sum_{i = 1}^{N} x_i - Nb \\
\frac{\d\epsilon(a, b)}{\d a} &=& 0 \\
0 &=& \sum_{i = 1}^{N} 2(y_i - ax_i - b)(-x_i) \\
0 &=& \sum_{i = 1}^{N} y_ix_i - a\sum_{i = 1}^{N} x_i^2 - b\sum_{i=1}^{N} x_i \\
\end{eqnarray*}

Now if we substitute:
\begin{eqnarray*}
s_x &=& \sum_{i = 1}^{N} x_i \\
s_y &=& \sum_{i = 1}^{N} y_i \\
s_{xy} &=& \sum_{i = 1}^{n} x_iy_i \\
s_{x^2} &=& \sum_{i = 1}^{N} x_i^2 \\
\end{eqnarray*}

we get the following system of equalities:
\[
\begin{array}{|ccccc}
  s_x a &+& Nb &=& s_y \\
  s_{x^2} a &+& s_xb &=& s_{xy} \\
\end{array}
\]

We can solve it using Cramer's rule:
\begin{eqnarray*}
\Delta_0 &=& \begin{Vmatrix} s_x & N \\ s_{x^2} & s_x \end{Vmatrix} \\
\Delta_1 &=& \begin{Vmatrix} s_y & N \\ s_{xy} & s_x \end{Vmatrix} \\
\Delta_2 &=& \begin{Vmatrix} s_x & s_y \\ s_{x^2} & s_{xy} \end{Vmatrix} \\
a &=& \Delta_1 / \Delta_0 \\
b &=& \Delta_2 / \Delta_0 \\
\end{eqnarray*}

\subsection*{2.2}

Implementation is in \texttt{fit\_line} and \texttt{test\_fit\_line}.

\end{document}
